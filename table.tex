\renewcommand{\baselinestretch}{0.8}
\newcommand{\PreserveBackslash}[1]{\let\temp=\\#1\let\\=\temp} 
\let\PBS=\PreserveBackslash
\begin{table}[htbp]
\centering
\begin{minipage}{0.89\textwidth}
\caption[Major Differences Between Neural Networks and Cell Signaling Networks]{Major Differences Between Neural Networks and Cell Signaling Networks\protect\footnote{You can also have footnote in table as in figure.}}
\vskip .25in
\label{table:NN_DIFF}

  \footnotesize
    \begin{tabular}
	{|p{0.04 \textwidth}
	|p{0.15 \textwidth}
	|p{ 0.35 \textwidth}
	|p{ 0.35 \textwidth}|} \hline
	\rule{0pt}{1pt} & & & \\
	\strut & 
		\PBS\raggedright{\em Salient features of networks} &
		\PBS\raggedright{\em Neural Networks} &
		\PBS \raggedright{\em Cell Signaling Networks}\\
	\rule{0pt}{1pt}& & & \\ \hline
	\rule{0pt}{1pt}& & & \\
	\strut (i)  & 
		\PBS\raggedright{nodes} & 
		\PBS\raggedright{all nodes typically identical} &
		\PBS\raggedright{nodes are not all equivalent in performance} \\
	\rule{0pt}{1pt}& & & \\ \hline
	\rule{0pt}{1pt}& & & \\
	\strut (ii) & 
		\PBS\raggedright{structure} &
		\PBS\raggedright{typically layered with feed-forward connections only} &
		\PBS\raggedright{regulatory signals often give rise to feedback connections resulting in cycles} \\
	\rule{0pt}{1pt}& & & \\ \hline
	\rule{0pt}{1pt}& & & \\
	\strut (iii) & 
		\PBS\raggedright{connectivity} & 
		\PBS\raggedright{highly connected} &
		\PBS\raggedright{more sparsely connected} \\
	\rule{0pt}{1pt}& & & \\ \hline
	\rule{0pt}{1pt}& & & \\
	\strut (iv) &  
		\PBS\raggedright{learning rule} &
		\PBS\raggedright{connectivity altered to perform a single function e.g.,~via a back-propagation algorithm} &
		\PBS\raggedright{cells must be able to respond to multiple stimuli effectively;changes typically occur via evolutionary processes} \\

	\rule{0pt}{1pt}& & & \\ \hline
    \end{tabular}
\\
\end{minipage}
\end{table}
\renewcommand{\baselinestretch}{1.65}
